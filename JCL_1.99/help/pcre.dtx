
@@pcre

Summary

Header conversions and dynamic library loading routines for pcre.dll
(Perl-compatible Regular Expressions).

Description

Header conversions and dynamic library loading routines for pcre.dll
(Perl-compatible Regular Expressions).

Requires pcre.dll for the Windows platform.

Requires libpcre.so.0 for UNIX and Linux platforms.

Use the compiler define PCRE_LINKONREQUEST to switch between static and
dynamic linking.

{$DEFINE PCRE_LINKONREQUEST}

It is set to dynamic by default. To disable the compiler define, simply
insert a '.' before the '$' character.

Note: If you enable static linking of a DLL, this means that the
pcre.dll *must* be in the users path or an AV will occur at startup.

See Also

JclPCRE

@@JclPCRE

Summary

Contains an implementation of a Perl-compatible Regular
Expression engine.

Description

Contains an implementation of a Perl-compatible Regular
Expression engine.

Requires the header conversions and dynamic library loading
code in pcre.pas.

Requires resource strings in JclResources.pas.

Requires pcre.dll. The latest release of PCRE is always
available using FTP from
ftp://ftp.csx.cam.ac.uk/pub/software/programming/pcre/.

During unit initialization, the pcre.dll dynamic link library
is loaded and mapped for use by the Perl-compatible Regular
Expression engine.

During unit finalization, the pcre.dll dynamic link library
is unloaded.

See Also

pcre


@@EPCREError

Summary

Exception raised an error occurs loading or accessing the PCRE.DLL
library.

Description

EPCREError represents an exception raised an error occurs loading the
PCRE.DLL library, or when a result code is returned from PCRE.DLL that
represents an error. EPCREError allows applications to use exception
handling to trap the exceptions specific to the JclPCRE implementation.

EPCREError extends its' ancestor class by adding the ErrorCode property
and the CreateRes method.

EPCREError exceptions can be raised when the PCRE.DLL library cannot be
loaded because it is not installed on the local computer, or when it
cannot be located in the search path at run-time.

EPCREError may also be raised when executing methods or accessing
properties in TJclAnsiRegEx that result in numeric return values from
PCRE.DLL representing an error.

<table>
Method                  Description
----------------------  -----------------------------------------------
TJclAnsiRegEx.Compile   Raised when the regular expression argument is
                          an empty string ('').
TJclAnsiRegEx.Captures  Raised when the Capture at the requested index
                          cannot be accessed using PCRE.DLL.
</table>

EPCREError exception messages use string values located in
JclResources.pas that represent the ErrorCode for the instance.

See Also

EPCREError.CreateRes,
EPCREError.ErrorCode,
TJclAnsiRegEx.Compile,
TJclAnsiRegEx.Captures


@@EPCREError.ErrorCode

Summary:

Represents the numeric value for the exception.

Description:

ErrorCode is a read-only property that represents the numeric value for
the exception. ErrorCode is updated using an argument value passed to
the CreateRes constructor.

Use Message to access the string containing the description of the
exception.

See Also:

Message,
EPCREError.CreateRes


@@EPCREError.CreateRes@PResStringRec@Integer

Parameters:

ResStringRec - String to use as the \description of the error message.

ErrorCode - Numeric value for the exception.

Summary:

Constructor for the object instance.

Description:

CreateRes is an alternate constructor for the exception.  Arguments
passed to the constructor are used to set the value for the ErrorCode
and Message properties in the object instance.

See Also:

Message,
EPCREError.ErrorCode


@@TJclAnsiRegEx

Summary

Implements a regular expression engine based on the Perl-compatible
Regular Expression library.

Description

TJclAnsiRegEx implements a regular expression engine based on the
Perl-compatible Regular Expression library.

TJclAnsiRegEx requires the header conversions and import routines in
pcre.pas, and the PCRE.DLL dynamic link library.

TJclAnsiRegEx provides properties and methods that act as wrappers to
routines in the PCRE.DLL library.

Use Compile to examine, study, and localize a regular expression.

Use Match to compare a text subject against the compiled regular
expression.

Use CaptureCount to determine the total number of captured character
classes for the regular expression.

Use Captures to access captured character classes by their
ordinal position.

Use CaptureOffset to access the starting and ending positions for
captured character classes by their ordinal position.

Use ErrorMessage to access the description of an error returned from
the PCRE.DLL library.

Use ErrorOffset to determine the offset into a regular expression where
an error was detected by the PCRE.DLL library.

Use Options read or write values that affect the configuration and
behavior of the PCRE.DLL library.

For more information about using TJclAnsiRegEx, please refer to the
topic:

* <link JclPCRE_Using, Using the JCL PCRE Classes>

See Also

Compile,
Match,
CaptureCount,
Captures,
CaptureOffset,
ErrorMessage,
ErrorOffset,
Options,
JclPCRE_Using


@@TJclAnsiRegEx.Create

Summary

Constructor for the object instance.

Description

Create is the Constructor for the object instance.

Create calls the inherited constructor, and allocates the
internal buffer size required for results from the PCRE library.

See Also

Destroy


@@TJclAnsiRegEx.Compile@AnsiString@Boolean@Boolean

Parameters

Pattern - Regular expression to use when comparing a text subject using
the regular expression engine.

Study - Indicates if optimization is required for the regular
expression.

UserLocale - Indicates if non-standard localization is required for the
regular expression.

Returns

Boolean - True indicates successful completion of the method.

Summary

Converts a regular expression into the form required for the PCRE.DLL
library.

Description

Compile is a method used to convert a regular expression into the form
required for the PCRE.DLL library, and to optionally perform
optimization and localization for the regular expression.

Pattern indicates the regular expression to be compiled. Pattern must
contain a string that meets the syntax and semantics of the regular
expressions supported by PCRE.  The power of regular expressions comes
from the ability to include alternatives and repetitions in the
pattern. These are encoded in the pattern by the use of metacharacters,
which do not stand for themselves but instead are interpreted in some
special way.

There are two different sets of metacharacters: those that are
recognized anywhere in the pattern except within square brackets, and
those that are recognized in square brackets. Outside square brackets,
the metacharacters are as follows:

<table>
Metacharacter   Description
--------------  ------------------------------------------------------
\\              general escape character with several uses
\^              assert start of string (or line, in multiline mode)
\$              assert end of string (or line, in multiline mode)
\.              match any character except newline (by default)
\[              start character class definition
\|              start of alternative branch
\(              start subpattern
\)              end subpattern
\?              extends the meaning of ( also 0 or 1 quantifier
                  also quantifier minimizer
\*              0 or more quantifier
\+              1 or more quantifier
                  also "possessive quantifier"
\{              start min/max quantifier
</table>

Part of a pattern that is in square brackets is called a "character
class". In a character class the only metacharacters are:

<table>
Metacharacter   Description
--------------  ------------------------------------------------------
\\              general escape character
\^              negate the class, but only if the first character
\-              indicates character range
\[              POSIX character class (followed by POSIX syntax)
\]              terminates the character class
</table>

Please refer to the documentation in <link pcrepattern.html>PCRE
Patterns</link> for a more detailed description of regular expressions
and metacharacters.

Study indicates if the regular expression should be inspected for
additional information that can be extracted to speed up matching
performance. Set Study to True if the same compiled regular expression
will be used in multiple calls to the Match method.

UserLocale indicates that a non-standard locale is in use on the local
machine, and show be used to override the character tables built into
the PCRE library. Set UserLocale to True to force the users' locale to
be used instead of the default encodings in the PCRE library.

Values in the Options property are used to configure the regular
expression engine in the PCRE library, and to alter the run-time
behavior of pattern matching. Set values in the Options property prior
to calling Compile or Match to control the configuration and behavior
of the PCRE library.

Refer to the documentation for TJclAnsiRegExOption for a description of
the values used in the Options property.

The compiled regular expression representing Pattern is stored
internally in TJclAnsiRegEx for subsequent use in the Match method. An
EPCREError exception is raised if Pattern contains an empty string
('').

Compile returns a Boolean value that indicates if the regular
expression in Pattern is successfully compiled (and optionally
optimized).

Use the ErrorMessage and ErrorOffset properties to determine the type
and location of a syntax error detected in the Pattern argument.

Use the Match method to comapre a text subject using the compiled
regular expression.

Exceptions

EPCREError - Exception raised when the regular expression in Pattern
contains an empty string ('').

See Also

Options,
Match,
ErrorMessage,
ErrorOffset,
TJclAnsiRegExOption,
TJclAnsiRegExOptions,
EPCREError


@@TJclAnsiRegEx.Destroy

Summary

Destructor for object instance.

Description

Destroy is the Destructor for object instance.

See Also

Create


@@TJclAnsiRegEx.Match@AnsiString@Cardinal

Parameters

Subject - Values to compare to the regular expression.

StartOffset - Offset in Subject to begin the comparison.  Default value
is 1.

Returns

Boolean - True when values in Subject match the regular expression.

Summary

Examines a text subject for values that match a compiled regular
expression.

Description

Match is a method used to compare the text subject specified in Subject
to the compiled Perl-compatible regular expression using algorithms in
the PCRE library.

StartOffset is an offset into Subject where the comparison should be
started.  The default value for StartOffset is 1, and indicates the
first character in the Delphi string data type.

Match returns a Boolean value that indicates if any of the text in
Subject matches the compiled regular expression. Match can return False
if there is no compiled regular expression (from the Compile method)
for the comparison, or when the value in Subject is an empty string
('').

Values in the Options property are used to configure the regular
expression engine in the PCRE library, and to alter the run-time
behavior of pattern matching. Set values in the Options property prior
to calling Compile or Match to control the configuration and behavior
of the PCRE library.

Refer to the documentation for TJclAnsiRegExOption for a detailed
description of the values used in the Options property.

Use Compile to create the compile regular expression for use in the
Match method.

The set of strings that result from comparison to a regular expression
are represented using the Captures property. Values in Captures are
accessed using the ordinal position of character groups or classes from
the regular expression.  Use CaptureCount to determine the number of
strings resulting from the comparison.  Use CaptureOffset to access the
starting and ending offsets into the Subject where captured character
classes were detected in the Match method.

See Also

Compile,
Captures,
CaptureOffset,
CaptureCount,
Options,
TJclAnsiRegExOption,
TJclAnsiRegExOptions


@@TJclAnsiRegEx.CaptureCount

Summary

Number of strings in the result for a comparison performed using the
Match method.

Description

CaptureCount is a read-only Integer property that represents the number
of strings in the result for a comparison performed using the Match
method.

The value in CaptureCount is updated when Match returns True.  Its'
minimum value should be 1 (for a literal match involving no
metacharacters or metacharacter classes) up to the number of strings
required to represent the metacharacters and metacharacter classes in
the regular expression.

CaptureCount can also be used to determine the range of ordinal
positions allowed when accessing indexed values in the Captures and
CaptureOffset properties.  The range of values is guaranteed to be in
the range 0 up to CaptureCount-1.

Use Captures to access a result string by its' ordinal position.

Use CaptureOffset to access the starting and ending offsets into the
text subject by its' ordinal position.

See Also

Match,
Compile,
Captures,
CaptureOffset


@@TJclAnsiRegEx.CaptureOffset

Summary

Indicates the location of a string in a text subject resulting from a
call to the Match method.

Description

CaptureOffset is an read-only TJclAnsiCaptureOffset property that
indicates the location of a string in a text subject resulting from a
call to the Match method. CaptureOffset is an indexed property, and
provides access to each string in the result by its' ordinal position.

Use CaptureCount to determine the total number of strings in the result
from the Match method. The range of values is in CaptureOffset and
Captures is guaranteed to be in the range 0 up to CaptureCount-1.

Use Captures to access the strings resulting from a call to the Match
method.

See the documentation in TJclAnsiCaptureOffset for a detailed
description of the members in the structure.

See Also

CaptureCount,
Captures,
Match,
Compile,
TJclAnsiCaptureOffset


@@TJclAnsiRegEx.Captures

Summary

Strings resulting from a call to the Match method.

Description

Captures is a read-only AnsiString property that provides access to
strings resulting from a call to the Match method. Captures is an
indexed property that provides access to each string in the result set
by its' ordinal position. Each string in Captures corresponds to a
match for metacharacters and character classes specifed in the compiled
regular expression.

Use CaptureOffset to access the starting and ending offsets of the
Capture string in the text subject.

Use CaptureCount to determine the total number of strings in the result
from the Match method. The index values used to access the Captures and
CaptureOffset properties is guaranteed to be in the range 0 up to
CaptureCount-1.

Note: The size of the AnsiString returned fro the Capture property is
limited to 1025 characters (the size of the buffer allocated for string
retrieval).

Exceptions

EPCREError - Raised when an index value exceeds the number of strings
in the captured results.

See Also

CaptureCount,
CaptureOffset,
Match,
EPCREError


@@TJclAnsiRegEx.ErrorMessage

Summary

Represents an error message generated when compiling a regular
expression.

Description

ErrorMessage is a read-only AnsiString property that represents an
error message generated when compiling a regular expression.

ErrorMessage represents the inital error found in the semantics or
syntax of the regular expression passed as an argument to the Compile
method.

ErrorMessage may contain the following:

<table>
Value   Meaning
------  ------------------------------------------------------------
0       no error
1       \\ at end of pattern
2       \\c at end of pattern
3       unrecognized character follows \\
4       numbers out of order in {} quantifier
5       number too big in {} quantifier
6       missing terminating ] for character class
7       invalid escape sequence in character class
8       range out of order in character class
9       nothing to repeat
10      operand of unlimited repeat could match the empty string
11      internal error: unexpected repeat
12      unrecognized character after (?
13      POSIX named classes are supported only within a class
14      missing )
15      reference to non-existent subpattern
16      erroffset passed as NULL
17      unknown option bit(s) set
18      missing ) after comment
19      parentheses nested too deeply
20      regular expression too large
21      failed to get memory
22      unmatched parentheses
23      internal error: code overflow
24      unrecognized character after (?<
25      lookbehind assertion is not fixed length
26      malformed number or name after (?(
27      conditional group contains more than two branches
28      assertion expected after (?(
29      (?R or (?digits must be followed by )
30      unknown POSIX class name
31      POSIX collating elements are not supported
32      this version of PCRE is not compiled with PCRE_UTF8 support
33      spare error
34      character value in \x{...} sequence is too large
35      invalid condition (?(0)
36      \\C not allowed in lookbehind assertion
37      PCRE does not support \\L, \\l, \\N, \\U, or \\u
38      number after (?C is > 255
39      closing ) for (?C expected
40      recursive call could loop indefinitely
41      unrecognized character after (?P
42      syntax error after (?P
43      two named subpatterns have the same name
44      invalid UTF-8 string
45      support for \\P, \\p, and \\X has not been compiled
46      malformed \\P or \\p sequence
47      unknown property name after \\P or \\p
48      subpattern name is too long (maximum 32 characters)
49      too many named subpatterns (maximum 10,000)
50      repeated subpattern is too long
51      octal value is greater than \\377 (not in UTF-8 mode)
</table>

Use ErrorOffset to locate the offset into the regular expression string
where the error is located.

Note: ErrorMessage and ErrorOffset are not related to EPCREError
exception messages generated at run-time by the PCRE library.

See Also

ErrorOffset,
Compile


@@TJclAnsiRegEx.ErrorOffset

Summary

Represents the location of an error generated when compiling a regular
expression.

Description

ErrorOffset is a read-only Integer property that represents the offset
into a regular expression string where an error has been detected.
ErrorOffset represents the location of the inital error found in the
semantics or syntax of the regular expression passed as an argument to
the Compile method.

Use ErrorMessage to get the description of the error in the regular
expression string.

Note: ErrorMessage and ErrorOffset are not related to EPCREError
exception messages generated at run-time by the PCRE library.

See Also

ErrorMessage,
Compile


@@TJclAnsiRegEx.Options

Summary

Flags for configuring and controlling the run-time behavior of the
regular expression engine.

Description

Options is a TJclAnsiRegExOptions property that represents the set of
option flags for configuring and modifying the behavior of the regular
expression engine.  Options can contain a set of values from the
TJclAnsiRegExOption enumeration.

See the document in TJclAnsiRegExOption for a detailed description of
values in the enumeration.

Values in the Options property are passed to the PCRE library during
execution of the Compile and Match methods. Set values in the Options
property prior to calling the Compile or Match methods.

See Also

TJclAnsiRegExOptions,
TJclAnsiRegExOption,
Compile,
Match


@@TJclAnsiCaptureOffset

Summary

Represents the starting and ending offsets for captured string results
in the TJclAnsiRegEx class.

Description

TJclAnsiCaptureOffset is a record type with members that represent the
starting and ending offsets for captured string results in the
TJclAnsiRegEx class.

TJclAnsiCaptureOffset instances are allocated and updated when reading
the indexed values in the TJclAnsiRegEx.Captures property.

Note: At the present time, the values in TJclAnsiCaptureOffset members
represent zero-based offsets.  Since were dealing with Delphi strings
in the result(s), this is likely to be changed to Pascal-style
string offsets where the initial offset into the string is 1 instead of
0.

See Also

TJclAnsiCaptureOffset.FirstPos,
TJclAnsiCaptureOffset.LastPos,
TJclAnsiRegEx,
TJclAnsiRegEx.Captures


@@TJclAnsiCaptureOffset.FirstPos

Summary

Description

FirstPos represents the starting offset into a text subject for a
captured result string.

See Also

TJclAnsiRegEx.Captures.
TJclAnsiRegEx.CaptureOffset


@@TJclAnsiCaptureOffset.LastPos

Summary

Description

LastPos represents the ending offset into a text subject for a
captured result string.

See Also

TJclAnsiRegEx.Captures.
TJclAnsiRegEx.CaptureOffset


@@TJclAnsiRegExOption

Summary

Represents bit flags used to configure or modify the PCRE regular
expression engine.

Description

TJclAnsiRegExOption is an enumerated type that represents bit flags
that can be used to configure or modify the behavior of the PCRE
regular expression engine.  Values in TJclAnsiRegExOption affect the
execution of the TJclAnsiRegEx.Compile and TJclAnsiRegEx.Match methods.

TJclAnsiRegExOption include the following values (and their PCRE
equivalents):

<table>
Value             PCRE                  Meaning
----------------  --------------------  -------------------------------
roIgnoreCase      PCRE_CASELESS         Do caseless matching
roMultiLine       PCRE_MULTILINE        \^ and \$ match new lines
                                          within data
roDotAll          PCRE_DOTALL           \. matches anything including
                                          the new lines
roExtended        PCRE_EXTENDED         Ignore whitespace and comments
roAnchored        PCRE_ANCHORED         Force pattern anchoring
roDollarEndOnly   PCRE_DOLLAR_ENDONLY   \$ not to match newline at end
roExtra           PCRE_EXTRA            PCRE extra features
roNotBOL          PCRE_NOTBOL           First character of the subject
                                          string is not the beginning
                                          of a line
roNotEOL          PCRE_NOTEOL           End of the subject string is
                                          not the end of a line
roUnGreedy        PCRE_UNGREEDY         Invert greediness of
                                          quantifiers
roNotEmpty        PCRE_NOTEMPTY
roUTF8            PCRE_UTF8             Run in UTF-8 mode
</table>

Values in TJclAnsiRegExOption can be assigned to the
TJclAnsiRegExOptions set type as used in the TJclAnsiRegEx.Options
property.

Note: PCRE must be built with UTF-8 support in order to use roUTF8.

See Also

TJclAnsiRegEx.Compile,
TJclAnsiRegEx.Match,
TJclAnsiRegEx.Options


@@TPCREIntArray
<unfinished>

Summary

Description

See Also


@@PPCREIntArray
<unfinished>

Summary

Description

See Also


@@TJclAnsiRegExOptions

Summary

Represents the set of bit flags to use for configuring and
controlling the TJclAnsiRegEx regular expression engine.

Description

Represents the set of TJclAnsiRegExOption bit flags to use for
configuring and controlling the TJclAnsiRegEx regular expression
engine.

See Also

TJclAnsiRegEx,
TJclAnsiRegExOption


@@JclPCRE_Intro
<title JclPCRE Introduction>

Summary

An introduction to the JCL implementation of a Perl-compatible Regular
Expression engine.

Description

The JCL (JEDI Code Library) contains an implementation of a
Perl-compatible Regular Expression engine in the file pas. This
unit requires the header conversions and dynamic library loading code
availbe in the file pcre.pas. It also requires resource strings defined
in the file JclResources.pas.

JclPCRE requires pcre.dll. The latest release of PCRE is always
available using FTP from
ftp://ftp.csx.cam.ac.uk/pub/software/programming/pcre/. pcre.dll must
be available in the search path on any machine that uses the JCLs' pcre
wrapper classes.

See Also

JclPCRE_Using, JclPCRE, pcre


@@JclPCRE_Using
<title Using the JCL PCRE Classes>
<autolink on>

Summary

Using the JCL PCRE Classes.

Description

The JCL PCRE classes provide access to functions and structures in
pcre.dll through the use of TJclAnsiRegEx and TJclAnsiRegExOptions
classes.

TJclAnsiRegEx is a class that provides properties and methods
that act as wrappers for routines and structures found in the
pcre.dll dynamic link library.

TJclAnsiRegEx is a non-visual object; it cannot be used on
the design surfaces of an IDE. It does not, however, require
any installation other that being availble in the search path
for the compiler. Simply add the pas unit to the
"uses" clause in your application.

<code>
  uses
    Classes, Windows, SysUtils, Forms, Dialogs,
    ActnList, ComCtrls, StdCtrls, Controls,
    JclPCRE;
</code>

In your program, you would normally allocate an instance of
TJclAnsiRegEx and configure its' options in the
TJclAnsiRegEx.Options property.

<code>
  var
    RE: TJclAnsiRegEx;
    REO: JclAnsiRegExOptions;
  ...

  RE := TJclAnsiRegEx.Create;
  REO := [roIgnoreCase, roMultiLine, roUnGreedy];
  RE.Options := REO;
</code>

TJclAnsiRegEx.Options is used primarily to control the behavior of the
regular expression engine during calls to the TJclAnsiRegEx.Compile and
TJclAnsiRegEx.Match methods. Values in the JclAnsiRegExOptions set
reflect the attributes passed or retrieved using routines in pcre.dll.
See JclAnsiRegExOptions and JclAnsiRegExOption for a detailed
\description of values.

Use the TJclAnsiRegEx.Compile method to specify the
Perl-compatible regular expression to use when matching a
text subject. TJclAnsiRegEx.Compile is essentially a wrapper
around the pcre_compile(), pcre_compile2(), pcre_study(), and
pcre_maketables() functions in pcre.dll.

See the documentation for the TJclAnsiRegEx.Compile method
for more details about arguments to the method.

Perl-compatible Regular Expressions are very flexible and
very powerful. With all of that utility comes some
complexity. Please refer to the pcrepattern documentation for
a detailed description of the syntax and semantics of
Perl-compatible Regular Expressions.

Use the TJclAnsiRegEx.ErrorMessage and
TJclAnsiRegEx.ErrorOffset properties to examine errors
detected when compiling the regular expression.

Use the TJclAnsiRegEx.Match method to compare the compiled
regular expression against a given subject string using a
matching algorithm that is similar to Perl's. The
TJclAnsiRegEx.Match method is a wrapper around the
pcre_exec() function in pcre.dll.

TJclAnsiRegEx.Match returns a boolean value to indicate that
elements of the regular expression exist in the subject text.
Use the TJclAnsiRegEx.CaptureCount property to find the
number of matching strings found for the regular expression.
Use the TJclAnsiRegEx.Captures property to access the string
values by their ordinal position. Use the
TJclAnsiRegEx.CaptureOffset property to access the offsets
into the subject text where the string match was located.

<code>
  // look for HTML anchor with HREF attribute
  RE.Compile('\<a\\s+href\\s*=\\s*(["'])?(.*)(["'])?(.*)\>\\s*(.*)\\s*\<\\/a\>',
    False, False);

  if not RE.Match(memoHTML.Lines.Text) then
  begin
    MessageDlg('No matches found', mtInformation, [mbOK], 0);
  end
  else
  begin
    ShowMessage('Found: ' +
      Copy(memoHTML.Lines.Text, RE.CaptureOffset[0].FirstPos,
        RE.CaptureOffset[0].LastPos - RE.CaptureOffset[0].FirstPos + 1));
  end;
</code>

TJclAnsiRegEx may raise exceptions when using its' properties
and methods. These exceptions can normally be handled in your
application code by responding to EPCREError exception
instances.

See Also

TJclAnsiRegEx,
TJclAnsiRegExOptions,
TJclAnsiRegExOption,
EPCREError,
pcrepattern,
pcre_compile,
pcre_exec


