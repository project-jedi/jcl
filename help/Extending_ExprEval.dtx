
$Id: Extending_ExprEval.dtx,v 1.1 2001/05/04 17:05:30 gothmog Exp $

@@$ExtExprEval
<title Extending the Expression Evaluator System>
The expression evaluator can be extended in several ways, depending on
your needs. Before trying to extend the evaluator, it is useful to
understand <link ExtExprEval_DataFlow, how data flows through the system>;
read this topic first.
<p>
Different extension ideas could be:
  * <link ExtExprEval_LibContext, Create a library context> that can be
    added to multiple evaluators.
  * <link $ExtExprEval_CustomSymbol, Create a symbol with custom effects>.
  * Use the compiling parser and
    <link ExtExprEval_AddEvalSystem, add an evaluation system>.
  * Use the evaluation system in a
    <link ExtExprEval_AddParseSystem, different parser>.


@@ExtExprEval_DataFlow
  <group $ExtExprEval>
  <title Data Flow in the Expression Evaluator system>
  Brief: This topic describes how data flows through the expression
    evaluator system.

  * Data Flow Outline *

    A string expression is fed into a <link TExprLexer, lexical analyser>.
    The job of the lexical analyser (lexer) is to break up this string
    expression into component parts - operators, identifiers and numbers.
    <p>
    Methods of the lexer are called by a parser (TExprEvalParser,
    TExprCompileParser). The parser drives the system - it pulls tokens
    from the lexer and either returns a concrete value (TExprEvalParser)
    or constructs an expression node (TExprCompileParser).
    <p>
    When a parser meets an identifier, it looks it up using a context
    object. The <link TExprContext, context> object is responsible for
    mapping an identifier to a <link TExprSym, symbol object>. A little
    more on symbol objects later.
    <p>
    If the parser is an evaluating parser, it returns a concrete
    floating-point value, and returns to whatever called it - job done.
    The situation is a little more interesting with a compiling parser,
    however. To compile expressions, the expression is first converted into
    an intermediate representation - a graph of expression nodes. This
    graph is directed (i.e. there is a direction in the relationship between
    nodes) and acyclic (it isn't possible to follow relationships from a node
    n and arrive back at node n through any positive number of intermediate
    nodes); in other words, it is a DAG.
    <p>
    The expression is converted into a DAG by calling a
    <link TExprNodeFactory, node factory object's> methods. The job of
    creating DAG nodes is devolved to a factory object to allow TExprNode
    descendants to be customized to create code for a particular
    implementation strategy.
    <p>
    The created nodes are stored in the node factory, in a linear list.
    After all nodes have been created, graph rewriting patterns may be
    applied to the DAG to perform optimizations like common sub-expression
    elimination (CSE), constant sub-expression evaluation and strength
    reduction (like converting constant divides to multiply by inverse),
    etc.
    <p>
    A concrete descendant of the node factory will have a GenCode method
    (which takes in some object capable of code generation) which iterates
    over all its internal nodes, telling each of them to generate code for
    the passed in object.
    <p>
    The job of the factory is now done; depending on the implementation
    strategy, the code generator may also be the code executor, or further
    steps may be needed to put it all together.


  * Different classes of classes *

    It follows from the above outline that the system's classes can be sorted
    into classes of their own:

    * <b>Lexer</b>: Lexical analysis is performed by a concrete descendant of
      TExprLexer. This is usually TExprSimpleLexer.

    * <b>Parser</b>: Parsing is performed by an object that implements a
      grammar and takes in a lexical source and context as parameters. Because
      the Parser drives the system, it doesn't necessarily have an abstract
      ancestor. Two main classes: TExprEvalParser (evaluating parser) and
      TExprCompileParser (compiling parser).

    * <b>Context</b>: Context (symbol lookup) is performed by a concrete
      descendant of TExprContext, usually a TExprSetContext containg multiple
      TExprSetContext and TExprHashContext instances.

    * <b>Symbol</b>: Symbol evaluation / compiling is performed by a
      concrete descendant of TExprSym. Evaluation and compiling are combined
      into a single object for ease of implementing new symbols. Evaluation
      depends on a TExprEvalParser, compiling depends on a TExprCompileParser
      and TExprNodeFactory descendant, and both depend on a TExprLexer
      descendant. <i>All symbols have names like TExpr\<name\>Sym</i>.

    * <b>Node Factory</b>: Expression node construction, optimization and
      code generation is the responsibility of a node factory. A concrete
      descendant is TExprVirtMachNodeFactory, which constructs customized
      node classes descended from TExprVirtMachNode.

    * <b>Expression Node</b>: Expression nodes represent elements of
      an expression, in a format that is suitable for optimization and
      code generation without a restriction to a particular compiled
      format. <i>All of the provided virtual machine's nodes have names
      like TExpr\<name\>VmNode</i>.

    * <b>Generated Code</b>: Depending on the implementation strategy,
      different methods of generating code and different generated code
      objects will be produced. For the provided virtual machine,
      code operations descend from TExprVirtMachOp. <i>All of the
      provided virtual machine's operations have names like
      TExpr\<name\>VmOp</i>.


  * The Shell Objects *

    The evaluator shell objects (TEvaluator, TCompiledEvaluator and
    TExpressionCompiler) orchistrate connections between component parts of
    the evaluation system. They use different components to achieve their
    respective properties.

    * TEvaluator uses a lexer (TExprSimpleLexer) and an evaluating parser
      (TExprEvalParser), with a user-provided context object.

    * TCompiledEvaluator uses a lexer (TExprSimpleLexer), a compiling
      parser (TExprCompileParser), a virtual machine (VM) (TExprVirtMach)
      and VM expression node factory (TExprVirtMachNodeFactory).

    * TExpressionCompiler uses a similar system to TCompiledEvaluator,
      but it maintains an internal list of created virtual machines,
      and creates a new virtual machine for every expression it compiles.

@@ExtExprEval_LibContext
  <group $ExtExprEval>
  <title Creating a library context>

  It is useful to create a context object that combines a library of
  often used and conceptually grouped symbols (functions, variables and
  constants), so that it support for these symbols can be added to an
  evaluator with minimum fuss. The easiest way to do it is like this:

  * Create a function that that constructs a context object. See Example 1.

  * Add this context to an evaluator's external context list. <b>Note that
    contexts added to an evaluator's external context list <i>are not</i>
    automatically freed</b>. You need to manage the context list's
    lifetime yourself; this is done so that multiple evaluators can use
    the same plug-in context. See Example 2.

    Example:
      <code>
      function TrigSin(x: TFloat64): TFloat64; // whatever

      function MakeTrigContext: TExprContext;
      begin
        // create context object with case insensitivity and
        // a hash size of 127
        Result := TExprHashContext.Create(False, 127);
        try

          // add functions; the correct symbol class must be chosen
          // to correspond to the function signature
          Result.Add(TExprFloat64FuncSym.Create('sin', TrigSin));

          // add constants
          // precision probably isn't important, but can be specfied
          // if desired anyway.
          Result.Add(TExprConstSym.Create('PI', Pi));

          // add other symbols similarly
        except
          Result.Free;
          raise;
        end;
      end;
      </code>

    Example:
      <code>
        var
          myEvaluator: TEvaluator;
          myTrigContext: TExprContext;
        // ...
          myTrigContext := MakeTrigContext;
          try
            myEvaluator.ExtContextSet.Add(MakeTrigContext);
            // ... use evaluator with trig stuff
            myEvaluator.ExtContextSet.Remove(MakeTrigContext);
          finally
            myTrigContext.Free;
          end;
      </code>

@@$ExtExprEval_CustomSymbol
  <group $ExtExprEval>
  <title Creating a custom symbol>

  An advantage the expression evaluation system has is that the way
  symbols are evaluated is not fixed; each symbol can individually
  communicate with the lexer and parser (which may be customized
  themselves) to produce a result (a floating-point value or an expression
  node) with any desired semantics.
  <p>
  To demonstrate the flexibility of the system, I'm going to develop
  a scoped namespace; a symbol will be added that, upon application of
  the dot '.' operator, will resolve a subsequent symbol within a
  smaller and unambiguous namespace.
  <p>
  For example, following on from the
  <link ExtExprEval_LibContext, Trig Context Library> article, we could
  set up our symbols so they are parsed like this:
    <code>
      'trig.sin(trig.pi * x / 180)'
    </code>
  while both of these would produce a syntax error:
    <code>
      'sin(pi * x / 180)'
      'trig.sin(pi * x / 180)'
    </code>
  <p>
  Steps:
    * <link ExtExprEval_CustomSymbol_Interface, Creating the class interface>
      The layout of the symbol class, specifying its ancestor and private
      fields.
    * <link ExtExprEval_CustomSymbol_CreateDestroy, The constructor and destructor>
      The implementation of the constructor and destructor.
    * <link ExtExprEval_CustomSymbol_GetNested, The helper GetNestedSymbol implementation>
      The implementation of the Evaluate method, for evaluating parsers.
    * <link ExtExprEval_CustomSymbol_EvalImpl, The Evaluate implementation>
      The implementation of the Evaluate method, for evaluating parsers.
    * <link ExtExprEval_CustomSymbol_CompileImpl, The Compile implementation>
      The implementation of the Compile method, for compiling parsers.

@@ExtExprEval_CustomSymbol_Interface
  <group $ExtExprEval_CustomSymbol>
  <title Creating the class interface>

  Here is the interface for our namespace symbol:
  <code>
    TExprNamespaceSym = class(TExprSym)
    private
      FContext: TExprContext;
      function GetNestedSymbol: TExprSym;
    public
      constructor Create(const AIdent: string; AContext: TExprContext);
      destructor Destroy; override;

      function Evaluate: TFloat; override;
      function Compile: TExprNode; override;

      property Context: TExprContext read FContext write FContext;
    end;
  </code>
  We descend from TExprSym, because that has everything we need (just
  a lexer, a couple of parsers and a node factory).
  <p>
  The private field FContext will hold the nested context; the namespace
  symbol will only act as a filter to this context, and won't manually
  resolve symbols itself.
  <p>
  The helper function GetNestedSymbol does some syntactic analysis using
  the lexer. It is a separate function because the job it does is needed for
  both evaluating and compiling.
  <p>
  The constructor takes in an identifier and a context. An identifier must
  be provided for all symbols, and be passed on to the inherited Create.
  The constructor also loads the internal context with the passed in context.
  <p>
  The destructor frees the internal context and calls the inherited
  destructor.
  <p>
  The Evaluate method reads in a '.' followed by an identifier, looks that
  identifier up in the internal context, and calls the found symbol's
  Evaluate method after setting up state. If no symbol is found, an
  exception is raised.
  <p>
  The Compile method reads in a '.' followed by an identifier, looks that
  identifier up in the internal context, and calls the found symbol's
  Compile method after setting up state. If no symbol is found, an
  exception is raised.
  <p>
  The Context property allows access to the symbol's internal context
  object.

@@ExtExprEval_CustomSymbol_CreateDestroy
  <group $ExtExprEval_CustomSymbol>
  <title The constructor and destructor>

  Here is the implementation of the constructor and destructor.
  <code>
    constructor TExprNamespaceSym.Create(const AIdent: string;
      AContext: TExprContext);
    begin
      inherited Create(AIdent);
      FContext := AContext;
    end;

    destructor TExprNamespaceSym.Destroy;
    begin
      FContext.Free;
      inherited Destroy;
    end;
  </code>
  It is important to call the inherited Create with the identifier
  for this symbol, otherwise the symbol will never be found, and it will
  conflict with any other unnamed symbol.
  <p>
  This namespace symbol owns its internal context, so the destructor frees
  it.

@@ExtExprEval_CustomSymbol_GetNested
  <group $ExtExprEval_CustomSymbol>
  <title The helper GetNestedSymbol implementation>

  Here is the helper method GetNestedSymbol.
  <code>
    function TExprNamespaceSym.GetNestedSymbol: TExprSym;
    begin
      // check for and skip '.'
      if Lexer.CurrTok <> etDot then
        raise Exception.Create('Expected: ''.''');
      Lexer.NextTok;

      // read an identifier
      if Lexer.CurrTok <> etIdentifier then
        raise Exception.Create('Expected: identifier');

      // look up identifier in internal namespace context
      Result := FContext.Find(Lexer.TokenAsString);
      if Result = nil then
        raise Exception.CreateFmt('Unknown identifier ''%s'' in namespace ''%s''',
          [Lexer.TokenAsString, Ident]);
    end;
  </code>
  This code demonstrates a dialog between the symbol and the lexer, doing
  syntactic analysis: expecting a dot '.' followed by an identifier.

@@ExtExprEval_CustomSymbol_EvalImpl
  <group $ExtExprEval_CustomSymbol>
  <title The Evaluate implementation>

  Here is the Evaluate method's implementation.
  <code>
    function TExprNamespaceSym.Evaluate: TFloat;
    var
      sym: TExprSym;

      oldLexer: TExprLexer;
      oldParser: TExprEvalParser;
    begin
      sym := GetNestedSymbol;

      // set up symbol's state
      oldLexer := sym.Lexer;
      oldParser := sym.EvalParser;
      try
        sym.Lexer := Lexer;
        sym.EvalParser := EvalParser;

        // get symbol's value
        Result := sym.Evaluate;
      finally
        sym.Lexer := oldLexer;
        sym.EvalParser := oldParser;
      end;
    end;
  </code>

  This sample shows how to set up a symbol's state for evaluation. It
  is important to save and restore the symbol's state; a symbol may be
  shared between evaluators (with different lexers etc.), and the
  evaluation of a symbol might rely on itself recursively.
  <p>
  The example (Example 1, on the top row above) shows how to interact
  with the parser to emualate array subscripting syntax.

  Example:
    This is an example of a different Evaluate method, which emulates
    array syntax, like 'a[3]', etc. It has an internal array pointer,
    FVarArray: PFloat32Array, which was passed in its constructor.

    <code>
      function TExprVar32ArraySym.Evaluate: TFloat;
      var
        idx: Integer;
      begin
        // check for and skip '['
        if Lexer.CurrTok <> etLBracket then
          raise Exception.Create('Expected: ''[''');
        Lexer.NextTok;

        // read index
        idx := Trunc(EvalParser.eval_expr);

        // check for and skip ']'
        if Lexer.CurrTok <> etLBracket then
          raise Exception.Create('Expected: '']''');
        Lexer.NextTok;

        // possibly do bounds-checking on array here

        // return array value
        Result := FVarArray^[idx];
      end;
    </code>

@@ExtExprEval_CustomSymbol_CompileImpl
  <group $ExtExprEval_CustomSymbol>
  <title The Compile implementation>

  Here is the implementation of the Compile method.
  <code>
    function TExprNamespaceSym.Compile: TExprNode;
    var
      sym: TExprSym;

      oldLexer: TExprLexer;
      oldParser: TExprCompileParser;
      oldNodeFactory: TExprNodeFactory;
    begin
      sym := GetNestedSymbol;

      // set up symbol's state
      oldLexer := sym.Lexer;
      oldParser := sym.CompileParser;
      oldNodeFactory := sym.NodeFactory;
      try
        sym.Lexer := Lexer;
        sym.EvalParser := EvalParser;
        sym.NodeFactory := NodeFactory;

        // compile symbol
        Result := sym.Compile;
      finally
        sym.Lexer := oldLexer;
        sym.CompileParser := oldParser;
        sym.NodeFactory := oldNodeFactory;
      end;
    end;
  </code>

  It is similar to the evaluate method, but it works with expression nodes
  instead of concrete values.
  <p>
  See the link to Example 1 (top row above) for a sample which demonstrates
  interaction with the compiling parser.

  Example:
    This is an example of a different Compile method, which emulates
    array syntax, like 'a[3]', etc. It has an internal array pointer,
    FVarArray: PFloat32Array, which was passed in its constructor.
    <p>
    <b>It isn't complete, because it needs support from the NodeFactory
    that it doesn't provide; it does provide an idea for
    extension, though...</b>

    <code>
      function TExprVar32ArraySym.Compile: TExprNode;
      var
        idx: TExprNode;
      begin
        // check for and skip '['
        if Lexer.CurrTok <> etLBracket then
          raise Exception.Create('Expected: ''[''');
        Lexer.NextTok;

        // read index
        idx := CompileParser.compile_expr;

        // check for and skip ']'
        if Lexer.CurrTok <> etLBracket then
          raise Exception.Create('Expected: '']''');
        Lexer.NextTok;

        // we need to truncate the index and load
        // an array value; this shows why compilation isn't as
        // flexible as evaluation: we need a new expression node
        // type, one to load a value from an array

        // I'm going to pretend I've added support:

        Result := NodeFactory.LoadVar32Array(FVarArray, idx);
      end;
    </code>

@@ExtExprEval_AddEvalSystem
  <group $ExtExprEval>
  <title Adding an evaluation subsystem>

  A separate evaluation subsystem can be implemented, but it does take
  a bit of work. Here is an outline of the tasks to be done:

  * Create a set of TExprNode descendants that are capable of some
    form of code generation; see the TExpr\<name\>VmOp classes in the
    unit source for examples.
  * Create a concrete TExprNodeFactory descendant that constructs TExprNode
    descendants in a way suitable for the desired evaluation subsystem
    strategy.
  * Add useful optimization methods and apply them, typically in the
    GenCode method.
  * Implement a GenCode method on the factory to spew out code into either
    a virtual machine or some code generator.

@@ExtExprEval_AddParseSystem
  <group $ExtExprEval>
  <title Adding a parsing subsystem>

  A parsing system can be added, but it isn't as flexible as adding an
  implementation system. The reason for this is that the way the system
  is currently implemented, symbols must be able to communicate with
  the lexer and parser. So, for evaluation they rely on TExprEvalParser
  and for compilation on TExprCompileParser. It isn't anticipated that
  adding a different parsing subsystem would be more useful than, say,
  adding a different evaluation system.
  <p>
  To accomodate attempts at custom parsers, the compile_expr and
  eval_expr methods are virtual; these are the methods that symbols
  usually call to do their work. The Evaluate and Compile methods
  are virtual too.
  <p>
  To accomodate custom lexers, the TExprLexer class is abstract, and the
  TExprToken type has 32 custom token codes. Use constants that typecast
  the token into a more readable name in this fasion:

  <code>
    const
      etStringLiteral = TExprToken(etUser0);
  </code>

  The lack of an explicit type specifier means that this is still a
  normal constant (i.e. never acts like an initialized variable),
  but the typecast means it is also typed.
